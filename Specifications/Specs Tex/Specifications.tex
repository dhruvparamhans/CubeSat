\documentclass[11pt,a4paper]{report}

\usepackage[margin=1.0in]{geometry}
\usepackage{polytechnique,url}
\usepackage{graphicx}
\usepackage{pdfpages}
\usepackage{multirow}
\usepackage{float}
\usepackage{titlesec}
\usepackage{subfig}
\title{Specifications for the Attitude Control and Determination System-XCubeSat} 
\author{Dhruv Sharma, Damien Seux, Quentin Lisack, Kevin Garanger}

\date{12 April 2015} 

\renewcommand{\titrecourt}{Specifications ADCS}
\setcounter{secnumdepth}{5}

\captionsetup{belowskip=12pt, aboveskip=4pt}
\begin{document}
\maketitle

\tableofcontents
\listoffigures
\listoftables

\chapter{Introduction}

The objective of this document is to summarize in one place the specifications related to the software of the Attitude Control and Determination System, henceforth ADCS of the nano-satellite X-CubeSat developed by the students of École Polytechnique, Paris. We have divided the document into the following sections 

\begin{itemize}
\item 
Definition of the architecture of the ADCS 
\item 
External Interfaces available on the ADCS 
\item 
Sensors and peripherals on the ADCS 
\item 
Data formats used for computing 
\item 
Protocols for testing 
\end{itemize}

This document shall be useful for all people currently concerned with the conception of the ADCS and eventually for students who will take over the project in the months to come. This document is however, far from being complete and additions, corrections and deletions will be made till the day of delivery. 

\chapter{Architecture of the ADCS}\thispagestyle{fancy}

The ADCS uses a microprocessor manufactured by ST Microelectronics based on the ARM Cortex-M4 Architecture. The model used is STM 32F405 RGT6

The available interfaces, sensors and other peripherals available on the ADCS are: 

\begin{itemize}
\item 
32 MB External Memory  
\item 
7(or 9) sun sensors 
\item 
Magnetometer
\item 
Gyrometer
\item 
H-Bridges to modulate power to the magneto-torquers 
\item 
A Serial interface with the On-Board Computer (\textit{Ordinateur de Bord} henceforth, ODB) 
\end{itemize}

Following is the complete plan for the ADCS: 

\includepdf[pages={-},angle=90]{QB50_ADCS_ST.pdf}

\chapter{Protocol for communication with ODB} \thispagestyle{fancy}

In this section, we define the protocol for communication between the ODB and the ADCS. The plan of the section is as follows. We first define the commands that the ODB will send to the ADCS and what these commands mean. Next we shall explain what happens in each of the threads of the ADCS. Finally we define how the ADCS will respond to each of the commands sent by the ODB. 

\section{Commands sent by the ODB} 

The commands sent by the ODB are as follows: 

	\begin{itemize}
		\item 
		\textbf{Ping}: The ODB sends a message to the ADCS to verify whether the ADCS is functioning or not
		\item 
		\textbf{Attitude Determination Mode}: ADCS moves into the attitude determination mode 
		\item 
		\textbf{Attitude Control Mode}: ADCS passes into attitude control mode 
		\item 
		\textbf{Raw Magnetic Field Values}: The ODB asks for the current raw magnetic field values 
		\item 
		\textbf{Raw Gyrometer Values}: The ODB asks for the current raw gyrometer values 
		\item 
		\textbf{Raw Sun Sensor Values}: The ODB asks for the current raw sun sensor values 
		\item 
		\textbf{True Magnetic Field Values}: The ODB asks for the current true magnetic field values 
		\item
		\textbf{True gyrometer values}: The ODB asks for the current true gyrometer values
		\item 
		\textbf{True sun vector values}: The ODB asks for the sun vector values 
		\item 
		\textbf{Set time}: The ODB sends a command to set the time for the ADCS. Serves to synchronize the time between the ADCS
							and the ODB 
		\item 
		\textbf{Read time}: The ODB calls for the current time of the ADCS. 
		\item 
		\textbf{STOP}: The ODB asks the ADCS to stop all its processes. 
	\end{itemize}
 
\chapter{Software Architecture} \thispagestyle{fancy}
This is the heart of this document. We discuss here the architecture of the software, the various peripherals and the sensors mounted on the microprocessors, the various functions associated with them detailing every time the input, output and the configuration parameters for each of the peripherals. 

We first begin by detailing how the software has been divided. Next we define the various classes that we have divided the software into. While the first division is an abstract division that we made to facilitate the development of the software, the structure of classes and their interdependance correspond to how the software will actually be implemented. 

Once the abstract division and the structure of classes are defined we detail the correspondence of the various ports/ pins available on the microprocessor and the peripherals connected. Next, we define the functions corresponding to each of the divisions. Irrespective of their eventual use, all functions shall find themselves in one of three categories that we shall define in the section \textbf{Division of the software}. 


\section{Division of the software} 

\begin{figure}[!h]
\centering 
\includegraphics[width=10cm]{arch_layers.jpg}
\caption{\textit{Division of the software in three layers each independent of the other}}
\label{layers_arch_soft}
\end{figure}
 
The ADCS software is divided into the following layers 

\begin{enumerate}
\item 
\textbf{API Firmware}: In this layer, all the low-level functionalities are collected. By low-level functionalities, we mean all those functionalities which are not directly relevant to the functioning of the algorithms. This layer doesn't provide any data to the algorithms. Rather, this layer is concerned only with initializing and configuring the various peripherals that we have on the ADCS.
\item 
\textbf{Driver}: jhis layer acts as a bridge between the lower level API firmware and the higher-level specialized functions that we require to run our algorithms. Ideally, in this layer we find the functions which get the raw data from the sensors, convert or calibrate them as necessary and send them to the algorithms for further use.
\item 
\textbf{Specialized Functions}: The specialized functions are the functions which implement the algorithms for determining and controlling the attitude of the satellite. These functions are high-level functions and require the data provided to them by the \textit{Driver} layer above to function. 
\end{enumerate}

\section{Structure of classes} 
Once we have abstracted the software as being divided into layers depending on the kind of functions we will write for the software, we consider next the division of the software into dedicated classes. The class structure as envisaged for the ADCS is summarized in the graph below. The arrow head signifies that the a particular class gets a particular piece of data from the class where the arrow originates. 

\begin{figure}[h]
\centering 
\includegraphics[scale=0.6]{class_structure}
\caption{\textit{Class structure for ADCS Software}}
\label{struct_class}
\end{figure}

The various layers as shown in the class structure above are detailed below 

\begin{enumerate}
\item 
\textbf{Sensors}: This layer corresponds to the API Firmware layer that we presented earlier. We have three types of sensors: Gyrometer, a magnetometer and 9 sun-sensors. This part of the software corresponds to the initialization and configuring the sensors (and the ports to which they are associated on the microprocessor) so that they can start generating the data. 
\item 
\textbf{Calculations 1}: This layer consists of a first layer of calculations effectuated once raw data is received from the sensors. In this layer, we receive the raw data from the sun-sensors, magnetometer and the gyrometer. 

\item 
\textbf{Calculations 2}:This corresponds to a supplementary layer of calculations which is effectuated once the raw data has been received (not shown in the class structure). This step might take different forms for different sensors. For instance, once we have received the sun sensor raw data, the supplementary steps include converting the raw data into a voltage and further combining the data from the 9 sun-sensors into one sun-vector with three components. For the magnetometer however, we can eventually consider calibrating the data with the temperature. Identically for the gyrometer as well. 

Finally, this step also relates to reading the terrestrial magnetic field map and the albedo map from the external memory. All this data once collected, calibrated and treated will be given to the algorithms, which we explain below. 

\item 
\textbf{Attitude Determination and Control Phase}: Once the data has been generated (after eventual calibrations), we have the layer where the actual control of the satellite takes place. The details of the algorithms will be presented in an ulterior section. 

\item
\textbf{PD-like controller}: This is what controls the satellite by providing us as output the magnetic moment to be generated in the magnetic coils. The controller then gives its output to the Coils-Attitude Control Interface. 

\item 
\textbf{Interface Coils-Attitude Control}: This layer corresponds to the calculations to be done to configure the magneto-torquers. The magneto-torquers are controlled by Pulse-Width Modulation(PWM). For PWM, we require the frequency of modulation (in KHz) and the duty cycle (in percent). The result of the algorithms, however, is the magnetic moment to be generated by the magnetic torquers. Thus we require precise data on the behaviour of the magnetic coils used to convert this magnetic moment into a duty-cycle. (The frequency of modulation shall be fixed for all the coils)

\item 
\textbf{Magnetic Coils}: This is the layer which corresponds to the actual control that will be applied via the magnetic coils. The coils are powered via PWM for a fixed frequency usually around 15-20KHz. 
\end{enumerate}


\section{Definition of ports}

In this section we define the various ports, what their functionalities are and to which of the peripherals they are connected. 

We also specify the alternate functions that need to be configured for each of the ports. These alternate functions could be Serial Communication (USART) for communicating with the ODB or an Analog to Digital Converter (ADC) for the sun-sensors. 

\begin{table}[H]
%\centering
\resizebox{1.15\textwidth}{!}{%
\begin{tabular}{ccccccc}
\hline
\multicolumn{1}{|c|}{Peripheral} & \multicolumn{1}{c|}{\begin{tabular}[c]{@{}c@{}}Name of\\   the port\end{tabular}} & \multicolumn{1}{c|}{Port on MPU} & \multicolumn{1}{c|}{Logical State} & \multicolumn{1}{c|}{\begin{tabular}[c]{@{}c@{}}Alternate\\ Function(AF)\end{tabular}} & \multicolumn{1}{c|}{AF Details 1} & \multicolumn{1}{c|}{AF Details 2} \\ \hline
\multicolumn{1}{|c|}{} & \multicolumn{1}{c|}{} & \multicolumn{1}{c|}{} & \multicolumn{1}{c|}{} & \multicolumn{1}{c|}{} & \multicolumn{1}{c|}{} & \multicolumn{1}{c|}{} \\ \hline
\multicolumn{1}{|c|}{\multirow{5}{*}{\begin{tabular}[c]{@{}c@{}}Sun \\ Sensor\end{tabular}}} & \multicolumn{1}{c|}{GS5} & \multicolumn{1}{c|}{PC0} & \multicolumn{1}{c|}{N/A} & \multicolumn{1}{c|}{ADC} & \multicolumn{1}{c|}{ADC\_Resolution\_8b} & \multicolumn{1}{c|}{Channel10} \\ \cline{2-7} 
\multicolumn{1}{|c|}{} & \multicolumn{1}{c|}{GS4} & \multicolumn{1}{c|}{PC1} & \multicolumn{1}{c|}{N/A} & \multicolumn{1}{c|}{ADC} & \multicolumn{1}{c|}{ADC\_Resolution\_8b} & \multicolumn{1}{c|}{Channel 11} \\ \cline{2-7} 
\multicolumn{1}{|c|}{} & \multicolumn{1}{c|}{GS3} & \multicolumn{1}{c|}{PC5} & \multicolumn{1}{c|}{N/A} & \multicolumn{1}{c|}{ADC} & \multicolumn{1}{c|}{ADC\_Resolution\_8b} & \multicolumn{1}{c|}{Channel 15} \\ \cline{2-7} 
\multicolumn{1}{|c|}{} & \multicolumn{1}{c|}{GS2} & \multicolumn{1}{c|}{PB0} & \multicolumn{1}{c|}{N/A} & \multicolumn{1}{c|}{ADC} & \multicolumn{1}{c|}{ADC\_Resolution\_8b} & \multicolumn{1}{c|}{Channel 8} \\ \cline{2-7} 
\multicolumn{1}{|c|}{} & \multicolumn{1}{c|}{GS1} & \multicolumn{1}{c|}{PB1} & \multicolumn{1}{c|}{N/A} & \multicolumn{1}{c|}{ADC} & \multicolumn{1}{c|}{ADC\_Resolution\_8b} & \multicolumn{1}{c|}{Channel 9} \\ \hline
 &  &  &  &  &  &  \\ \hline
\multicolumn{1}{|c|}{\multirow{6}{*}{PWM}} & \multicolumn{1}{c|}{F1} & \multicolumn{1}{c|}{PB14} & \multicolumn{1}{c|}{N/A} & \multicolumn{1}{c|}{TIM1} & \multicolumn{1}{c|}{Output Compare} & \multicolumn{1}{c|}{} \\ \cline{2-7} 
\multicolumn{1}{|c|}{} & \multicolumn{1}{c|}{R1} & \multicolumn{1}{c|}{PB13} & \multicolumn{1}{c|}{0 or 1} & \multicolumn{1}{c|}{N/A} & \multicolumn{1}{c|}{} & \multicolumn{1}{c|}{} \\ \cline{2-7} 
\multicolumn{1}{|c|}{} & \multicolumn{1}{c|}{F2} & \multicolumn{1}{c|}{PB12} & \multicolumn{1}{c|}{0 or 1} & \multicolumn{1}{c|}{N/A} & \multicolumn{1}{c|}{} & \multicolumn{1}{c|}{} \\ \cline{2-7} 
\multicolumn{1}{|c|}{} & \multicolumn{1}{c|}{R2} & \multicolumn{1}{c|}{PB11} & \multicolumn{1}{c|}{N/A} & \multicolumn{1}{c|}{TIM2} & \multicolumn{1}{c|}{Output Compare} & \multicolumn{1}{c|}{} \\ \cline{2-7} 
\multicolumn{1}{|c|}{} & \multicolumn{1}{c|}{F3} & \multicolumn{1}{c|}{PB10} & \multicolumn{1}{c|}{N/A} & \multicolumn{1}{c|}{TIM2} & \multicolumn{1}{c|}{Output Compare} & \multicolumn{1}{c|}{} \\ \cline{2-7} 
\multicolumn{1}{|c|}{} & \multicolumn{1}{c|}{R3} & \multicolumn{1}{c|}{PA5} & \multicolumn{1}{c|}{0 or 1} & \multicolumn{1}{c|}{N/A} & \multicolumn{1}{c|}{} & \multicolumn{1}{c|}{} \\ \hline
 &  &  &  &  &  &  \\ \hline
\multicolumn{1}{|c|}{\multirow{2}{*}{\begin{tabular}[c]{@{}c@{}}Comm\\ ODB\end{tabular}}} & \multicolumn{1}{c|}{TX\_ADCS} & \multicolumn{1}{c|}{PA10} & \multicolumn{1}{c|}{N/A} & \multicolumn{1}{c|}{USART1\_RX} & \multicolumn{1}{c|}{Baud Rate} & \multicolumn{1}{c|}{Bit size} \\ \cline{2-7} 
\multicolumn{1}{|c|}{} & \multicolumn{1}{c|}{RX\_ADCS} & \multicolumn{1}{c|}{PA9} & \multicolumn{1}{c|}{N/A} & \multicolumn{1}{c|}{USART1\_TX} & \multicolumn{1}{c|}{Baud Rate} & \multicolumn{1}{c|}{Bit size} \\ \hline
 &  &  &  &  &  &  \\ \hline
\multicolumn{1}{|c|}{MISC} & \multicolumn{1}{c|}{\begin{tabular}[c]{@{}c@{}}On/Off\\   Pont H\end{tabular}} & \multicolumn{1}{c|}{PB14} & \multicolumn{1}{c|}{0 or 1} & \multicolumn{1}{c|}{N/A} & \multicolumn{1}{c|}{} & \multicolumn{1}{c|}{} \\ \hline
\end{tabular}
}
\caption{List of peripherals with alternate functions and other configurable parameters}
\label{list_all_periph}
\end{table}

\section{API Firmware} 

This section details all the functions concerning the initialization and configuration of the peripherals. 

Some generalities are in order. The initialization of any port goes through the following steps:

\begin{enumerate}
\item 
We first enable the peripheral clock corresponding to the sensor in question.
\item 
We then enable the GPIO clock corresponding to the sensor
\item 
We then enable the alternate function for the GPIO if required 
\item 
We then configure the GPIO for use as an alternate function 
\item 
We initialize the GPIO using its init function
\item 
Configure the functionalities for the alternate function 
\item 
Finally initialize the alternate function using the corresponding init function. 
\end{enumerate}

One of the most important steps in the initialization of any port is to initialize the clock connected to the port. Each port is connected to one of two buses. We resume the various clocks connected to the various ports to serve as a reference to the reader. 

% Please add the following required packages to your document preamble:
% \usepackage{multirow}
% \usepackage{graphicx}
\begin{table}[H]
\centering
\resizebox{0.5\textwidth}{!}{%
\begin{tabular}{ccc}
\hline
\multicolumn{1}{|c|}{Clock} & \multicolumn{1}{c|}{Peripheral} & \multicolumn{1}{c|}{RCC Peripheral} \\ \hline
\multicolumn{1}{l}{} & \multicolumn{1}{l}{} & \multicolumn{1}{l}{} \\ \hline
\multicolumn{1}{|c|}{\multirow{13}{*}{APB1}} & \multicolumn{1}{c|}{TIM2} & \multicolumn{1}{c|}{RCC\_APB1Periph\_TIM2} \\ \cline{2-3} 
\multicolumn{1}{|c|}{} & \multicolumn{1}{c|}{TIM3} & \multicolumn{1}{c|}{RCC\_APB1Periph\_TIM3} \\ \cline{2-3} 
\multicolumn{1}{|c|}{} & \multicolumn{1}{c|}{TIM4} & \multicolumn{1}{c|}{RCC\_APB1Periph\_TIM4} \\ \cline{2-3} 
\multicolumn{1}{|c|}{} & \multicolumn{1}{c|}{TIM5} & \multicolumn{1}{c|}{RCC\_APB1Periph\_TIM5} \\ \cline{2-3} 
\multicolumn{1}{|c|}{} & \multicolumn{1}{c|}{TIM6} & \multicolumn{1}{c|}{RCC\_APB1Periph\_TIM6} \\ \cline{2-3} 
\multicolumn{1}{|c|}{} & \multicolumn{1}{c|}{TIM7} & \multicolumn{1}{c|}{RCC\_APB1Periph\_TIM7} \\ \cline{2-3} 
\multicolumn{1}{|c|}{} & \multicolumn{1}{c|}{TIM12} & \multicolumn{1}{c|}{RCC\_APB1Periph\_TIM12} \\ \cline{2-3} 
\multicolumn{1}{|c|}{} & \multicolumn{1}{c|}{TIM13} & \multicolumn{1}{c|}{RCC\_APB1Periph\_TIM13} \\ \cline{2-3} 
\multicolumn{1}{|c|}{} & \multicolumn{1}{c|}{TIM14} & \multicolumn{1}{c|}{RCC\_APB1Periph\_TIM14} \\ \cline{2-3} 
\multicolumn{1}{|c|}{} & \multicolumn{1}{c|}{SPI2} & \multicolumn{1}{c|}{RCC\_APB1Periph\_SPI2} \\ \cline{2-3} 
\multicolumn{1}{|c|}{} & \multicolumn{1}{c|}{SPI3} & \multicolumn{1}{c|}{RCC\_APB1Periph\_SPI3} \\ \cline{2-3} 
\multicolumn{1}{|c|}{} & \multicolumn{1}{c|}{USART2} & \multicolumn{1}{c|}{RCC\_APB1Periph\_USART2} \\ \cline{2-3} 
\multicolumn{1}{|c|}{} & \multicolumn{1}{c|}{USART3} & \multicolumn{1}{c|}{RCC\_APB1Periph\_USART3} \\ \hline
\multicolumn{1}{l}{} & \multicolumn{1}{l}{} & \multicolumn{1}{l}{} \\ \hline
\multicolumn{1}{|c|}{\multirow{11}{*}{APB2}} & \multicolumn{1}{c|}{TIM1} & \multicolumn{1}{c|}{RCC\_APB2Periph\_TIM1} \\ \cline{2-3} 
\multicolumn{1}{|c|}{} & \multicolumn{1}{c|}{TIM8} & \multicolumn{1}{c|}{RCC\_APB2Periph\_TIM8} \\ \cline{2-3} 
\multicolumn{1}{|c|}{} & \multicolumn{1}{c|}{USART1} & \multicolumn{1}{c|}{RCC\_APB2Periph\_USART1} \\ \cline{2-3} 
\multicolumn{1}{|c|}{} & \multicolumn{1}{c|}{USART6} & \multicolumn{1}{c|}{RCC\_APB2Periph\_USART6} \\ \cline{2-3} 
\multicolumn{1}{|c|}{} & \multicolumn{1}{c|}{ADC} & \multicolumn{1}{c|}{RCC\_APB2Periph\_ADC} \\ \cline{2-3} 
\multicolumn{1}{|c|}{} & \multicolumn{1}{c|}{ADC1} & \multicolumn{1}{c|}{RCC\_APB2Periph\_ADC1} \\ \cline{2-3} 
\multicolumn{1}{|c|}{} & \multicolumn{1}{c|}{ADC2} & \multicolumn{1}{c|}{RCC\_APB2Periph\_ADC2} \\ \cline{2-3} 
\multicolumn{1}{|c|}{} & \multicolumn{1}{c|}{ADC3} & \multicolumn{1}{c|}{RCC\_APB2Periph\_ADC3} \\ \cline{2-3} 
\multicolumn{1}{|c|}{} & \multicolumn{1}{c|}{TIM9} & \multicolumn{1}{c|}{RCC\_APB2Periph\_TIM9} \\ \cline{2-3} 
\multicolumn{1}{|c|}{} & \multicolumn{1}{c|}{TIM10} & \multicolumn{1}{c|}{RCC\_APB2Periph\_TIM10} \\ \cline{2-3} 
\multicolumn{1}{|c|}{} & \multicolumn{1}{c|}{TIM11} & \multicolumn{1}{c|}{RCC\_APB2Periph\_TIM11} \\ \hline
\multicolumn{1}{l}{} & \multicolumn{1}{l}{} & \multicolumn{1}{l}{} \\ \hline
\multicolumn{1}{|c|}{\multirow{8}{*}{AHB1}} & \multicolumn{1}{c|}{GPIOA} & \multicolumn{1}{c|}{RCC\_AHB1\_GPIOA} \\ \cline{2-3} 
\multicolumn{1}{|c|}{} & \multicolumn{1}{c|}{GPIOB} & \multicolumn{1}{c|}{RCC\_AHB1\_GPIOB} \\ \cline{2-3} 
\multicolumn{1}{|c|}{} & \multicolumn{1}{c|}{GPIOC} & \multicolumn{1}{c|}{RCC\_AHB1\_GPIOC} \\ \cline{2-3} 
\multicolumn{1}{|c|}{} & \multicolumn{1}{c|}{GPIOD} & \multicolumn{1}{c|}{RCC\_AHB1\_GPIOD} \\ \cline{2-3} 
\multicolumn{1}{|c|}{} & \multicolumn{1}{c|}{GPIOE} & \multicolumn{1}{c|}{RCC\_AHB1\_GPIOE} \\ \cline{2-3} 
\multicolumn{1}{|c|}{} & \multicolumn{1}{c|}{GPIOF} & \multicolumn{1}{c|}{RCC\_AHB1\_GPIOF} \\ \cline{2-3} 
\multicolumn{1}{|c|}{} & \multicolumn{1}{c|}{GPIOG} & \multicolumn{1}{c|}{RCC\_AHB1\_GPIOG} \\ \cline{2-3} 
\multicolumn{1}{|c|}{} & \multicolumn{1}{c|}{GPIOH} & \multicolumn{1}{c|}{RCC\_AHB1\_GPIOH} \\ \hline
\end{tabular}
}
\caption{Clocks connected to each of the peripherals}
\label{clock_periph}
\end{table}
The configuration and initialization of all the functionalities presented below will follow the list of steps detailed. 

\subsection{PWM} 

\subsubsection{Initialization}

The following functions will be used for initialization purposes 

\begin{enumerate}
	\item 
		\textbf{Setting on the H-Bridge}
		
		Syntaxe: void setPWMON()
		
		This function switches on the H-bridge which in its turn control the coils. 

% Please add the following required packages to your document preamble:
% \usepackage{graphicx}
		\begin{table}[H]
			\centering
			\resizebox{0.6\textwidth}{!}{%
			\begin{tabular}{|c|c|c|}
				\hline
				Name of the port & Port on MPU & Logical State \\ \hline
				ON/OFF PONT H & PB14 & 1 \\ \hline
			\end{tabular}
	}
			\caption{PWM-ON}
			\label{on_pwm}
		\end{table}
\item 
	\textbf{Configuring the ports which determine the current direction}
	
	Syntaxe: void configDirs()
	
	\begin{table}[H]
\centering
\resizebox{0.6\textwidth}{!}{%
\begin{tabular}{|c|c|c|}
\hline
Name of the port & Port on MPU & \multicolumn{1}{l|}{} \\ \hline
R1 & PB13 & \multirow{3}{*}{\begin{tabular}[c]{@{}c@{}}To be configured as \\ General Purpose\\ Input/Output\end{tabular}} \\ \cline{1-2}
F2 & PB12 &  \\ \cline{1-2}
R3 & PA5 &  \\ \hline
\end{tabular}
}
\caption{Configure the ports for the direction}
\label{config_dirs}
\end{table}
	
\item 
	\textbf{Configure the ports which supply power}
	
	Syntaxe: void configTorquers()
	
	This function configures the ports connected to the coils to be used as Timers. 
	
	\begin{table}[H]
		\centering
		\resizebox{0.5\textwidth}{!}{%
		\begin{tabular}{|c|c|c|}
			\hline
			Name of the port & Port on MPU & \multicolumn{1}{l|}{} \\ \hline
			F1 & PB14 & TIM1 \\ \hline
			R2 & PB11 & \multirow{2}{*}{TIM2} \\ \cline{1-2}
			F3 & PB10 &  \\ \hline
\end{tabular}
}
\caption{Configure the ports as Timers}
\label{config_torquers}
\end{table}

\item 
	\textbf{Configure TIM1}
	
	Syntaxe: void configTIM1(uint32\_t PWM\_freq)
	
	Once the ports have been configured to be used as timers, we need to configure the timers to function at the frequency we require. The TIM1 is connected to the AHB1 Bus which has a clock frequency of 168MHz. Thus one first prescales the timer to be used as 

\item 
	\textbf{Configure TIM2}
	
	Syntaxe: void configTIM2(uint32\_t PWM\_freq)
	
\item 
	\textbf{Configure PWM1 }
	
	Syntaxe: void configPWM1(uint32\_t duty\_cycle)

\item 
	\textbf{Configure PWM2 }
	
	Syntaxe: void configPWM2(uint32\_t duty\_cycle)
	

\item 
	\textbf{Configure PWM3 }
	
	Syntaxe: void configPWM3(uint32\_t duty\_cycle)
\end{enumerate}

\subsubsection{Termination}

\paragraph{Configuring the ports}

We use the following ports to be used for providing power to the magnetic coils using PWM. The ports: \textbf{PB14, PB11, PB10} will be configured to be used as timers, whereas \textbf{PB13, PB12} and \textbf{PA5} will be used as general input/output ports. The last three ports will be used to determine the direction of the current in the coils. 

\begin{table}[H]
\centering
\resizebox{\textwidth}{!}{%
\begin{tabular}{|c|c|c|c|c|c|c|}
\hline
Peripheral & \begin{tabular}[c]{@{}c@{}}Name of\\   the port\end{tabular} & Port on MPU & Logical State & \begin{tabular}[c]{@{}c@{}}Alternate\\ Function(AF)\end{tabular} & AF Details 1 & AF Details 2 \\ \hline
\multirow{6}{*}{PWM} & F1 & PB14 & N/A & TIM1 & Output Compare &  \\ \cline{2-7} 
 & R1 & PB13 & 0 or 1 & N/A &  &  \\ \cline{2-7} 
 & F2 & PB12 & 0 or 1 & N/A &  &  \\ \cline{2-7} 
 & R2 & PB11 & N/A & TIM2 & Output Compare &  \\ \cline{2-7} 
 & F3 & PB10 & N/A & TIM2 & Output Compare &  \\ \cline{2-7} 
 & R3 & PA5 & 0 or 1 & N/A &  &  \\ \hline
\end{tabular}
}
\caption{List of PWM Ports} 
\label{list_pwm}
\end{table}






\subsection{ADC}


\subsection{USART} 

\section{Driver} 

\section{Specialized Functions} 

\chapter{Data formats}\thispagestyle{fancy}

\chapter{Testing}\thispagestyle{fancy} 

We define here the tests to be conducted before the launch of the CubeSat. We need to conduct these tests to guarantee the functioning of the software and the electronics used for the ADCS. We will thus test that each element of the ADCS is functional, is being powered correctly, is able to communicate with the ODB and the ADCS is reacting in a coherent fashion with the data that it is receiving. Precisely, the tests to be conducted are as follows: 


\begin{itemize}
\item 
Verify that the ADCS is being powered correctly and that it can be activated and deactivated by the ODB 
\item 
Verify that the magnetic field measured by the magnetometer is coherent. We will test this using a Helmholtz coil et compare the data received through the Debug port on the ADCS with the data measured by an independent magnetometer. 
\item 
Verify that the coils are powered correctly and that they can be powered on and off by the ADCS 
\item 
Tester how the coils react to the commands sent by the ADCS. In particular, we need to verify the orientation of the magnetic field generated by the coils(an error in the connection suffices to inverse the polarity of the field created by the coils), then verify that the norm of the field thus created is within 10 percent of the expected value. To this end, we will operate the ADCS using the ODB. 
\item 
Verify the coherence of the data received from the different sensors(gyrometers etc). To this end, we will test each of the sensors one by one. The magnetometer will be tested during test no 2, the gyrometer will be tested on a motorised 6-axis platform. Just as for the magnetometer, we will compare the data received through the Debug port on the ADCS with the movement imposed by the platform. 
\item 
The sun sensor will be tested by a light source in a dark room. 
\end{itemize}

These verification tests will be conducted during the following series of tests: 

\begin{itemize}
\item
First, we will conduct all these tests and use them as a reference for the next sequence of tests 
\item 
The protocol of the tests foresees conducting \textit{Pre Thermal Vacuum Tests}: the tests to be conducted before the satellite is exposed to thermal vacuum. We will use the results from the first series of tests 
\item 
The satellite will be then placed in thermal vacuum 
\item 
We then conduct \textit{Post Thermal Vacuum Tests}: This is identical to the 
\textit{Pre Thermal Vacuum Tests}. Thus, we will compare the results from the two series of tests and analyse the effect of thermal vacuum on the satellite 
\item 
Next, we will conduct tests in a vacuum chamber under extreme temperature conditions. We will conduct the 1st, 3rd and 5th series of tests at the maximum and minimum temperatures that the satellite is expected to endure during any particular orbit. 
\item 
Finally, we will conduct a final series of tests comprising of all the tests listed above to verify the functioning of the ADCS in its integrality
\end{itemize}

\chapter{Referentials}\thispagestyle{fancy} 

We define here the referentials used during the functioning of the ADCS. Many referentials are required to be defined: the referential of the satellite itself, the referential of the orbit, the referntial in which the attitude is measured, the referential linked to the centre of the Earth in which the position needs to be measured. 

The referentials chosen are: 

\begin{enumerate}
\item 
\textbf{Referential of the satellite}: The axes chosen are: 

\begin{itemize}
\item 
+Z pointing in the direction of the FIPEX Module 
\item 
+Y orthogonal to the axe of the battery opposite to the column of the satellite 
\item 
+X is chosen such that the coordinate system XYZ is a direct triad. 
\end{itemize}

\begin{figure}[!h]
\centering 
\includegraphics[width=5cm]{referential.jpg}
\caption{\textit{Definition of the coordinate axes for the referential of the satellite}}
\label{ref_satellite}
\end{figure}

\item 
\textbf{Earth Centered referential} :  The axes X and Y are defined to be in the equatorial plan whereby the X-axis points towards the International Reference Meridian (IRM) and the Z-Axis is pointing towards the International Reference Pole. 

\item 
\textbf{Orbital Referential}: This new reference frame corresponds to the flight reference frame. The attitude of the satellite is measured as the quaternion that transforms the orbital reference frame into the reference frame of the satellite, always in agreement with the ICD of the FIPEX. This reference frame is a local reference frame with the axes R, S and T. 

\begin{itemize}
\item 
R-axis points in the direction of motion 
\item 
S-axis points towards earth. 
\item 
T-axis is determined such that we have a direct orthonormal triad
\end{itemize}
\end{enumerate}

\end{document}